% chapter 5: Convection

\chapter{Convection}~\label{ch:convection}

The analysis of the ${}^{220}$Rn calibration data from XENON100~\cite{Aprile:2016pmc} revealed a buoyancy-driven convection pattern with a considerable $8\1{mm/s}$ fluid speed. A similar analysis from LUX~\cite{Malling:2014} of \Rn~in background data in revealed a pattern with the same features but a higher fluid speed of $3\1{cm/s}$. This was in stark contrast to the lack of convection that was observed in EXO-200~\cite{Albert:2015vma}.

\section{Convection in XENON1T}~\label{sec:convection}

Given the similarities in the convection patterns between XENON100 and LUX and the differences between how those experiments inject xenon into the active volume, concrete predictions of what to expect in XENON1T are impossible to make. Conceptually, we might expect a similar single-cell pattern, or the increased detector size might allow the formation of two cells. Thus, it is important that any attempt to measure the convection be as agnostic possible, in order to avoid projecting expecations onto the data.

To measure convection we require two decays out of a decay chain so that there is a meaningful correlation between decay vertices. We further impose a number of requirements on these two decays.
\begin{enumerate}
    \item One isotope in the chain must be capable of being injected into the detector and must mix throughout.
    \item The two decays of interest must be easily identifiable and shouldn't confuse position reconstruction in any way.
    \item The livetime of the second decay must be relatively short such that $v_{ave}\tau$ is larger than position reconstruction uncertainties, yet small enough such that within two or three livetimes it's still closer to its parent than any others (to facilitate easy matching).
    \item Any further activity in the decay chain must either decay away quickly or be removable (decay is preferable), unless this measurement is done at the end of the detector's lifetime, in which case this can be relaxed.
\end{enumerate}

Requirememt $(1)$ excludes all decay chains without some gasseous form, or that cannot form some gasseous compount (conceptually similar to CH$_3$T or UF$_6$). Requirement $(2)$ points towards isotopes that primarily have an $\alpha$ mode, as a high-energy $\gamma$ will scatter multiple times, which makes position reconstruction difficult. Isotopes with $\beta$-decays are difficult to identify due to the lack of defining spectral features. Requirement $(3)$ can be fulfilled by reducing the injected activity, but this then means the measurement requires more time. Furthermore, if the livetime of the second isotope is not short compared to the timescale of convection, this also makes matching difficult, as the effects of curvature of the fluid streamlines will need to be incorporated.

Pairings of radon and polonium are often ideal for this purpose.

\subsection{$^{220}$Rn-$^{216}$Po}

Given how successful the pairing of $^{220}$Rn and $^{216}$Po were for XENON100 for convection measurements, it is a logical place to start for XENON1T. However, as shown in Figure~\ref{fig:rn220}, the only a paucity of the $^{220}$Rn atoms injected into the active volume reach the drift region, with the majority decaying at or below the cathode.

\begin{figure}[htb]
    \includegraphics[width=\textwidth]{figures/rnveto/rn220}
    \caption{A plot of {\textit s1\_area\_fraction\_top} versus {\textit s1} for the alpha ROI using $^{220}$Rn calibration data. As can be seen, the majority of events are in the very bottom of the detector, with only a few reaching the drift region where convection can be measured.}\label{fig:rn220}
\end{figure}

\subsection{$^{222}$Rn-$^{218}$Po}

Another available option to match \Po~decays to their parent \Rn~decays. This is a far less trivial endeavour, as the $3.1\1{min}$ half-life of \Po~is quite lengthy, so even at a very modest $1\1{mm/s}$ convection speed, half of all polonium atoms will move 10s of cm away from their parents.

The \Rn~background rate is $14\1{\mu Bq/kg}$, for a total activity in the active volume of $28\1{mBq}$. The population of \Rn~necessary to provide this activity is easily found via $R = \frac{\dd N}{\dd t} = -\lambda N$, where $R$ is the rate, $\lambda = \ln 2/t_{1/2}$ the decay constant, and $N$ the number of atoms. Thus, we find the \Rn~population size is about $13\,300\1{atoms}$. These will mix uniformly throughout the detector, giving an average distance between \Rn~atoms to be $3.9\1{cm}$. We apply the same calculations for \Po~and find an average population of $7.5\1{atoms}$ and an average spacing of some $47\1{cm}$. The small population of \Po~means the normal Poissonian fluctuations will have a more significant impact than on \Rn.

\subsubsection{Decay identification}

Selection of \Rn~and \Po~is very straightforward. A plot of the reconstructed $z$-coordinate versus cS1 is shown in Figure~\ref{fig:z_cs1} for alpha decays.

\begin{figure}[htb]
    %\includegraphics[width=\textwidth]{figures/rnveto/z_cs1}
    \includegraphics[width=\textwidth]{figures/rnveto/dsdt}
    \caption{Reconstructed $z$ position versus corrected scintillation light for the alpha ROI. Note the breakdown of the correction factor towards the top and bottom as PMT saturation becomes significant. Four pouplations are evident in the bulk: $^{210}$Po, \Rn, \Po, and $^{214}$Po. The $^{214}$Po is often misreconstructed due to the topology of \BiPo~events deep in the detector.}\label{fig:z_cs1}
\end{figure}

The $^{210}$Po population exists nearly exclusively on the PTFE reflectors that form the wall of the TPC; even a moderate radial cut of only one or two cm significantly reduces its size. The $^{214}$Po band shows up much more clearly if one plots {\it s1\_area\_fraction\_top} rather than $z$, as the short half-life of $^{214}$Po causes some confusion for event reconstruction of the $^{214}$Bi-$^{214}$Po events.

As we are primarily interested in the center two populations, we can readily select them in the bulk by drawing a few curves on Figure~\ref{fig:z_cs1}, although this tends to break down close to the cathode or liquid surface as the populations smear together. Alternately, these populations can be easily clustered by machine learning algorithms, which provides some additional selection power in these edge regions (see Appendix~\ref{app:ml}).

\subsubsection{Decay matching}

Armed with reasonably clean populations of \Rn~and \Po, the work of decay matching can begin. To do this in as agnostic a fashion as possible, we begin by plotting the temporal and spatial separations of all \Rn~and all \Po~in Figure~\ref{fig:dsdt}. As we can see, most of this parameter space is dominated by random, incorrect pairings. Within 30 or 40 seconds after the \Rn~decay, its \Po~daughter merges with the bulk population and cannot be identified with an agnostic approach.

\begin{figure}[htb]
    \includegraphics[width=\textwidth]{figures/rnveto/dsdt}
    \caption{Spatial and temporal separations of \Rn~and \Po~events. The ``signal'' population of decay pairs begins at the origin and quickly merges with the ``background'' band of random pairs. Clearly, agnosticism is very difficult at times longer than about 30 seconds or distances further than about $10\1{cm}.}\label{fig:dsdt}
\end{figure}

Even though the 30 seconds of matching only gives $1-\exp \lambda_{\mathrm{Po}}t = 10\%$ of all possible decays, one day's worth of background data contains some $2\,500$ total \Rn~decays, or about $7\,500$ matched decays per month. Evenly spread over the active region, this is one match every $5\1{cm}$, which is sufficient to do a preliminary mapping of convection.

\subsection{Convection map}

By building a three-dimensional map of the displacement of each matched pair we see the shape of the pattern is, as in XENON100 and LUX, a single cell, with much lower speeds of around $3\1{mm/s}$. Figure~\ref{fig:convec} is a projection approximately along the axis of angular momentum.

\begin{figure}[htb]
    \includegraphics[width=\textwidth]{figures/rnveto/convection_full}
    \caption{The measured convection field in XENON1T. Blue arrows indicates upward movcement; red, downward. A single cell pattern, as was observed in XENON100 and LUX, is clearly seen. Typical speeds are $3\1{mm/s}$.}\label{fig:convec}
\end{figure}

We also see some interesting behavior in the corners. The lack of pairs in these regions indicate a few things. It is unlikely that there is no \Rn~there, but it is entirely possible that our event selection requirements perform very poorly here. This also might be interpreted as hints of counter-current eddies forming in the corners of the detector.

\section{Simulation}

As the manner of xenon injection in XENON100, LUX, and XENON1T are all different yet all three have the same convection pattern, we can conclude that recirculation does not have a significant impact on convection. Thus, a computational fluid dynamics (CFD) simulation can have some predictive power between experiments, as long as the thermodynamic boundary conditions are correctly modeled. Additionally, a convection field from simulation can probably reveal more about the underlying detector operation than measurements from data.

A variety of different software packages, commercial and otherwise, are available to do this kind of simulation. ANSYS Fluent~\cite{fluent} was used for this work on advice from professor Carlo Scalo of Purdue engineering, also because it is available on the Purdue computing clusters.

A simplified model of the active volume was used to build the simulation mesh, and thermodynamic data from NIST~\cite{NIST} was used at the operating conditions of the detector. The Boussinesq approximation~\cite{Boussinesq:1897} was used to model the density. As we expect laminar, buoyancy-driven flows, the approximation is valid. A representative result is shown in Figure~\ref{fig:cfd_sample}.

\begin{figure}[htb]
    %\includegraphics[width=\textwidth]{figures/rnveto/cfd_sample}
    \includegraphics[width=\textwidth]{figures/rnveto/BigSphere}
    \caption{Simulated convection pattern for XENON1T. A single cell is evident with speeds of a few mm/s. A simple temperature gradient is sufficient to drive this system, though the mechanisms that determine the direction of angular momentum is not yet understood. Counter-currents can be seen in the corners.}\label{fig:cfd_sample}
\end{figure}

Simulations for both steady-state and transient results were performed. The results agree qualitatively with each other, although vortex shedding from counter-currents in the corners are routinely observed.

\section{Comparison between simulation and data}

The simulation results agree qualitatively with the data. A simple temperature gradient is sufficient to establish and drive a single-cell convection pattern, although the symmetry-breaking mechanism that determines the orientation of the pattern is not yet understood. No correlation has been found between the layout of pipes to and from cryogenics and purification and the direction of cell angular momentum.

