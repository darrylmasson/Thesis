%
%  thesis.tex  2011-07-01  Mark Senn  http://engineering.purdue.edu/~mark
%
%  This is the thesis ``root file''.
%
%  To print the final copy of your thesis, put a '%'
%  in front of the \includeonly command and type
%  (from page 71 of _LaTeX User's Guide and Reference Manual_, 2nd edition):
%      latex thesis
%      bibtex thesis
%      latex thesis
%      latex thesis
%
%  In "Reference:" listings below:
%      KEY  MEANING
%      TM   ``A Manual for the Preparation of Graduate Theses'',
%             seventh revised edition, The Graduate School, 2006.
%             http://www2.itap.purdue.edu/gradschool//Publications/graduate-thesis-manual.pdf
%      PU   ``A Manual for the Preparation of Graduate Theses'',
%           The Graduate School, Purdue University, 1996.
%           http://www2.itap.purdue.edu/gradschool//Publications/graduate-thesis-manual.pdf
%
%  Search for "CHANGE" below and change things as necessary.
%  I recommend putting "%%" before any existing lines that
%  need to be changed and adding your new line(s) immediately
%  below the existing lines.
%

% See http://www.ecn.purdue.edu/~mark/puthesis/#Options
% for documentclass options.
% CHANGE NEXT LINE?
\documentclass[ece,dissertation]{puthesis}

% Define "align" environment used in demo-mathematics.tex.
% CHANGE NEXT LINE?
\usepackage{amsmath}

% Define "multicols" environment environment used in demo-multicols.tex.
% CHANGE NEXT LINE?
\usepackage{multicol}

% Define "subfigure" environment used in "demo-figure.tex".
% CHANGE NEXT LINE?
\usepackage{subfigure}

\usepackage{epstopdf}
\sloppy

% Title of thesis (used on cover and in abstract).
% The title shown must be the full, official title of the
% thesis.  Superscripts and subscripts are not permitted in
% the title.
% Reference: TM 26.
% Use \title{Put Title Here} for a one-line title.
% Use \\ to separate lines.
% Put % at the end of the last line to avoid getting an extra space
% in the abstract.
% There are two forms of title: one line or more than one line.
% There are examples of both below.
% Only use one \title.
% CHANGE NEXT FOUR LINES.
\title{An Example Thesis Done with LaTeX}
%\title{%
%  An Example Thesis Done with LaTeX\\
%  with a Very Long Title%
%}

% First author name with first name first is used for cover.
% Second author name with last name first is used for abstract.
% Your full name as it appears in the University records appears
% on the cover.
% Reference: TM 26, 29.
% There are two forms of author, with and without initials.
% There are examples of both below.
% Only use one \author line.
% CHANGE NEXT TWO LINES.
\author{Mark Senn}{Senn, Mark}
%\author{Mark D. Senn}{Senn, Mark D.}

% First is long title of degree (used on cover).
% Second is abbreviation for degree (used in abstract).
% Third is the month the degree was (will be) awarded (used on cover
% and abstract).
% Last is the year the degree was (wlll be) awarded (used on cover
% and abstract).
% The degree title for all doctoral candidates is ``Doctor of Philosophy.''
% The precise degree names for master's candidates appear in the list of
% ``Degrees Offered'' in the Graduate School bulletin.
% The date is the month and year that the degree is actually awarded.
% (If you have registered for ``degree only,'' revise the thesis title
% page to reflect the new date on which the degree is to be awarded.)
% Reference: TM 26--27, 30.
% CHANGE NEXT LINE?
\pudegree{Doctor of Philosophy}{Ph.D.}{May}{2007}

% Major professor (used in abstract).
% Use, for example:
%     \majorprof{John Q. Professor}
%     \majorprofs{John Q. Professor and Thomas R. Jones}
%     \majorprofs{John Q. Professor, Thomas R. Jones, and David S. Smith}
% depending on the number of major professors you have.
% CHANGE NEXT LINE.
\majorprof{John Q. Professor}

% Campus (used only on cover)
% Use one of the following:
%     Fort Wayne
%     Hammond
%     Indianapolis
%     West Lafayette
%     Westville
% Reference: TM 27.
% CHANGE NEXT LINE?
\campus{West Lafayette}

% My command definitions not specific to my thesis.
% CHANGE NEXT LINE?
\input{mydefs}


% My command definitions specific to my thesis.

% CHANGE NEXT LINE TWO LINES?
% Set things up so \margins will show where the margins on the page are.
\newcommand{\margins}{\Repeat{Show where the margins for the page are.}{4}}

% CHANGE NEXT TWO LINES?
% Let typing "\en" be exactly the same as typing "\ensuremath". 
\let\en=\ensuremath

% CHANGE NEXT FIVE LINES?
% Define a \ve command with two arguments, so if it called with
%     \ve an
% it will expand to
%     {\en{a_1},~\en{a_2},\ \ldots,~\en{a_{n}}}
\newcommand{\ve}[2]{\en{#1_1},~\en{#1_2},\ \ldots,~\en{#1_{#2}}}


% To LaTeX only some parts of your thesis put the
% names of the parts to include here.  For example,
% \includeonly{front} would only process front.tex.
% \includeonly{front,introduction} would only process
% front.tex and introduction.tex.
% To print the final copy of your thesis put a '%'
% in front of the \includeonly command and run LaTeX
% three times to make sure that all cross-references
% are correct.  Then run BibTeX once and LaTeX twice
% more.
% CHANGE NEXT LINE?
%\includeonly{front,introduction}

\begin{document}

% Start a new volume for your thesis.  All theses must have at least one
% volume.  If your thesis is too long to fit in one binder put another
% "\volume" between chapters below.
\volume

% Front matter (dedication, etc.).
% front matter

%\begin{document} 

 % Dedication page is optional.

  % Acknowledgements page is optional

  % The preface is optional.


  % The Table of Contents will be automatically created for you
  % using information you supply in
  %     \chapter
  %     \section
  %     \subsection
  %     \subsubsection
  % commands.
\tableofcontents

  % The List of Tables will be automatically created for you using
  % information you supply in
  %     \begin{table} ... \end{table}
  % environments.
\listoftables

  % The List of Figures will be automatically created for you using
  % information you supply in
  %     \begin{figure} ... \end{figure}
  % environments.
\listoffigures

  % List of Symbols is optional.
\begin{symbols}

$m$& mass\cr
$v$& velocity\cr

\end{symbols}

  % List of Abbreviations is optional.
\begin{abbreviations}

LXe& Liquid xenon\cr
GXe& Gasseous xenon\cr
TPC& Time projection chamber\cr
WIMP& Weakly interacting massive particle\cr
DM& Dark matter\cr
LNGS& Laboratori Nazionali del Gran Sasso

\end{abbreviations}

  % Nomenclature is optional.

  % Glossary is optional


  % Abstract is required.

\begin{abstract}

This is where the abstract goes.
Not much to say yet.

\end{abstract}

%\end{document}


% Put chapter \include commands here.
% CHANGE \include{...} COMMANDS BELOW?
%
%  revised  introduction.tex  2011-09-02  Mark Senn  http://engineering.purdue.edu/~mark
%  created  introduction.tex  2002-06-03  Mark Senn  http://engineering.purdue.edu/~mark
%
%  This is the introduction chapter for a simple, example thesis.
%


\chapter{Introduction}

This is the introduction.
The first paragraph after a heading is not indented.

This is a sentence.
This is a sentence.
This is a sentence.
This is a sentence.
This is a sentence.


\section{Section Heading}

This is a sentence.
This is a sentence.
This is a sentence.
This is a sentence.
This is a sentence.


\subsection{Subsection heading}

This is a sentence.
This is a sentence.
This is a sentence.
This is a sentence.
This is a sentence.


\subsubsection{Subsubsection heading}

This is a sentence.
This is a sentence.
This is a sentence.
This is a sentence.
This is a sentence.


% Summary and/or conclusions are optional but often used.
% The summary and/or conclusions often are the last
% major division(s) of the text.
% Reference: TM 32.
% CHANGE NEXT LINE?
%
%  summary.tex  2007-02-06  Mark Senn  http://www.ecn.purdue.edu/~mark
%

\chapter{Summary}

This is the summary chapter.


% Recommendations are optional.
% You may include recommendations as a major division if your
% subject matter and research dictate.
% Reference: TM 32.
% CHANGE NEXT LINE?

\chapter{Recommendations}

Buy low. Sell high.


% Appendices are optional.
% Appendices are not necessarily part of every thesis. Appendices are used
% for supplementary illustrative material, original data, computer programs,
% and other material not necessarily appropriate for inclusion within the
% text of your thesis. 
% Reference: TM 33.
% Use "\appendix" for one appendix or "\appendices" for more than one
% appendix.
% CHANGE NEXT 7 LINES?
\appendices
%
%  revised  demo-citations.tex  2011-09-02  Mark Senn  http://www.ecn.purdue.edu/~mark
%  created  demo-citations.tex  2007-03-21  Mark Senn  http://www.ecn.purdue.edu/~mark
%


\chapter{Demonstrate Citations}

I typed

\begin{verbatim}
    For \LaTeX\ answers I refer to
    % note to self: {\em \LaTeX: A Document Preparation System\/}
    \cite{Lamport:1994}
    and then to
    % note to self: {\em The \LaTeX\ Companion\/}
    \cite{Goossens:1994}
    or
    % note to self: {\em A Guide to LaTeX\/} (1999)
    \cite{Kopka:1999}.
    % note to self: {\em A Guide to LaTeX\/} (1999)
    \cite{Kopka:1999}
    is an updated edition of the 1995 edition
    \cite{Kopka:1995}.
\end{verbatim}

to get

\begin{quotation}
    For \LaTeX\ answers I refer to
    % note to self: {\em \LaTeX: A Document Preparation System\/}
    \cite{Lamport:1994}
    and then to
    % note to self: {\em The \LaTeX\ Companion\/}
    \cite{Goossens:1994}
    or
    % note to self: {\em A Guide to LaTeX\/} (1999)
    \cite{Kopka:1999}.
    % note to self: {\em A Guide to LaTeX\/} (1999)
    \cite{Kopka:1999}
    is an updated edition of the 1995 edition
    \cite{Kopka:1995}.
\end{quotation}

%
%  demo-figures.tex  2009-10-30  Mark Senn  http://engineering.purdue.edu/~mark
%
%  Demonstrate how to do figures.
%

\chapter{Demonstrate Figures}

The
\verb+h+
specifier used in all the examples below
tells \LaTeX\ to put the figure
``here''
instead of trying
to find a good spot
at the top or bottom of a page.
Specifiers can be combined, for example,
``\verb+\begin{figure}[htbp!]+''.

The complete list of specifiers:

\begin{center}
    \renewcommand{\baselinestretch}{1}\normalsize
    \begin{tabular}{ll}
        \bf Specifier& \bf Description\cr
        \tt b& bottom of page\cr
        \tt h& here on page\cr
        \tt p& on separate page of figures\cr
        \tt t& top of page\cr
        \tt !& try hard to put figure as early as possible\cr
    \end{tabular}
\end{center}

Label ``fi:not-centered'' is ``\ref{fi:not-centered}''.
Label ``sf:four-parts-c'' is ``\ref{sf:four-parts-c}''.

\Repeat{This is the first paragraph.}{5}

\begin{figure}[h]
  \includegraphics{plot.eps}
  \caption{%
    By default figures are not centered.
    This is a long caption to demonstrate that captions are single spaced.
  }
  \label{fi:not-centered}
\end{figure}

\Repeat{This is the second paragraph.}{10}

\begin{figure}[h]
  \centering
  \includegraphics{plot.eps}
  \caption{Use {\tt \char'134centering\/} to center figures.}
  \label{fi:centered}
\end{figure}

\Repeat{This is the third paragraph.}{15}

\begin{figure}[h]
  \centering
  \includegraphics{plot.eps}
  \caption{This is another figuure.}
  \label{fi:another}
\end{figure}

\Repeat{This is the fourth paragraph.}{10}

\begin{figure}[h]
  \centering 
  \subfigure[First subcaption.]{\label{sf:two-parts-a}  \includegraphics[width=0.3\textwidth]{plot.eps}}%
  \hskip 0.5truein
  \subfigure[Second subcaption.]{\label{sf:two-parts-b}\includegraphics[width=0.3\textwidth]{plot.eps}}
  \caption{This figure has two parts.}
  \label{fi:two-parts}
\end{figure}

\Repeat{This is the fifth paragraph.}{10}

\begin{figure}[h]
  \centering
  \subfigure[First subcaption.]{\label{sf:four-parts-a}  \includegraphics[width=0.3\textwidth]{plot.eps}}%
  \hskip 0.5truein
  \subfigure[Second subcaption.]{\label{sf:four-parts-b}\includegraphics[width=0.3\textwidth]{plot.eps}}
  \subfigure[Third subcaption.]{\label{sf:four-parts-c}\includegraphics[width=0.3\textwidth]{plot.eps}}%
  \hskip 0.5truein
  \subfigure[Fourth subcaption.]{\label{sf:four-parts-d}\includegraphics[width=0.3\textwidth]{plot.eps}}
  \caption{This figure has four parts.}
  \label{fi:four-parts}
\end{figure}

\Repeat{This is the sixth paragraph.}{10}

%
%  THIS FILE DOES SOME UNUSUAL THINGS TO MAKE
%  IT EASIER TO DO DEMONSTRATIONS.  IT SHOULD
%  NOT BE USED AS AN EXAMPLE OF HOW TO PREPARE
%  A FILE.  SEE THE OUTPUT OF THIS FOR LATEX
%  INPUT AND OUTPUT EXAMPLES.
%




%
%  demo-mathematics.tex  2008-12-09  Mark Senn  http://engineering.purdue.edu/~mark
%

\chapter{Demonstrate Mathematics}

    % Use single spacing.
    \Baselinestretch{1}

    % You don't normally need this.
    \mbox{}

    \begin{verbatim}
% From _More Math Into LaTeX_, 4th Edition, page 152:
%     TeX uses $$ to open and close a displayed math environment.
%     In LaTeX, this may occassionally cause problems.  Don't do it.
\[
    E = mc^2
\]
    \end{verbatim}
% From _More Math Into LaTeX_, 4th Edition, page 152:
%     TeX uses $$ to open and close a displayed math environment.
%     In LaTeX, this may occassionally cause problems.  Don't do it.
\[
    E = mc^2
\]
    \vskip\baselineskip
    \hrule
    \vskip0.5\baselineskip
    \filbreak

    \begin{verbatim}
\begin{equation}
    E = mc^2
\end{equation}
    \end{verbatim}
\begin{equation}
    E = mc^2
\end{equation}
    \vskip\baselineskip
    \hrule
    \vskip0.5\baselineskip
    \filbreak

    \begin{verbatim}
% Mydefs.tex defines \be to be \begin{equation} and
% \ee to be \end{equation}.
\be
    E = mc^2
\ee
    \end{verbatim}
% Mydefs.tex defines \be to be \begin{equation} and
% \ee to be \end{equation}.
\be
    E = mc^2
\ee
    \vskip\baselineskip
    \hrule
    \vskip0.5\baselineskip
    \filbreak

    \begin{verbatim}
\be
    x = -\frac{b}{2a} \pm \frac{\sqrt{b^2 - 4ac}}{2a}
\ee
    \end{verbatim}
\be
    x = -\frac{b}{2a} \pm \frac{\sqrt{b^2 - 4ac}}{2a}
\ee
    \vskip\baselineskip
    \hrule
    \vskip0.5\baselineskip
    \filbreak

    \begin{verbatim}
% requires \usepackage{amsmath}; use align* for no equation number
\begin{align}
    a = {}& b + c\\
    x = {}& y + z
\end{align}
    \end{verbatim}
% requires \usepackage{amsmath}; use align* for no equation number
\begin{align}
    a = {}& b + c\\
    x = {}& y + z
\end{align}
    \vskip\baselineskip
    \hrule
    \vskip0.5\baselineskip
    \filbreak

    \begin{verbatim}
\[
    Z = \left(
        \begin{array}{cc}
            a& b\\
            c& d
        \end{array}
    \right)
\]
    \end{verbatim}
\[
    Z = \left(
        \begin{array}{cc}
            a& b\\
            c& d
        \end{array}
    \right)
\]
    \vskip\baselineskip
    \hrule
    \vskip0.5\baselineskip
    \filbreak

    \begin{verbatim}
\begin{equation}
    \begin{split}
        a = {}& b + c\\
            {}& + d + e
    \end{split}      
\end{equation}
    \end{verbatim}
\begin{equation}
    \begin{split}
        a = {}& b + c\\
            {}& + d + e
    \end{split}      
\end{equation}
    \vskip\baselineskip
    \hrule
    \vskip0.5\baselineskip
    \filbreak

    \begin{verbatim}
\be
    (\cos x)^2 + (\sin x)^2 = 1
\ee
    \end{verbatim}
\be
    (\cos x)^2 + (\sin x)^2 = 1
\ee
    \vskip\baselineskip
    \hrule
    \vskip0.5\baselineskip
    \filbreak

    \begin{verbatim}
If $X = \cos x$ and $Y = \sin x$ then $X^2 + Y^2 = 1$.
    \end{verbatim}
If $X = \cos x$ and $Y = \sin x$ then $X^2 + Y^2 = 1$.
    \vskip\baselineskip
    \hrule
    \vskip0.5\baselineskip
    \filbreak

%
%  demo-multicols.tex  2007-03-19  Mark Senn  http://www.ecn.purdue.edu/~mark
%
%  Demonstrate multicols.
%
%  The multicols package must be loaded for this to work.
%  To load the multicols package put
%      \usepackage{multicols}
%  between the "\documentclass" and "\begin{document}" commands.
%

\chapter{Demonstrate Multicols}

% Put this amount of space between the columns.
\setlength{\columnsep}{0.5truein}

% Separate the columns with a vertical rule this wide.
\setlength{\columnseprule}{0.4pt}

\Repeat{This is one column.}{25}

\begin{multicols}{2}
\Repeat{This is two columns.}{25}
\end{multicols}

\begin{multicols}{3}
\Repeat{This is three columns.}{25}
\end{multicols}

\begin{multicols}{4}
\Repeat{This is four columns.}{25}
\end{multicols}

\begin{multicols}{5}
\Repeat{This is five columns.}{25}
\end{multicols}

%
%  demo-tables.tex  2013-03-29  Mark Senn  http://engineering.purdue.edu/~mark
%
%  Demonstrate how to do tables.
%

\chapter{Demonstrate Tables}

% \newlength{\ta}
% \newlength{\tb}
% \newlength{\tc}
% 
% \settowidth{\ta}{\vbox{\hbox{Money}\hbox{Market}}}
% \settowidth{\tb}{\vbox{\hbox{Stocks}\hbox{and}\hbox{Bonds}}}
% \settowidth{\tc}{\vbox{\hbox{Money}\hbox{Market}\hbox{and}\hbox{Stocks}}}
% 
% {
%     \renewcommand{\baselinestretch}{1}
%     \begin{table}
%       \caption{%
%         \hfil Allocation of the IRA and Keogh Wealth\hfil\break
%         \mbox{}\hfil for Investors With or Without Brokerage Accounts\hfil
%       }
%       \label{tab:ira}
%       \begin{center}
%         \begin{tabular}%
%           {%
%             |%
%             c%
%             |%
%             >{\centering\hspace{0pt}}m{\the\ta}%  Money Market
%             |%
%             c%                                    Stocks 
%             |%
%             c%                                    Bonds
%             |%
%             c%                                    Diversified
%             |%
%             >{\centering\hspace{0pt}}m{\the\tb}%  Stocks and Bonds
%             |%
%             >{\centering\hspace{0pt}}m{\the\tc}%  Money Market and Stocks
%             |%
%             c%                                    Others
%             |%
%           }
%           \hline
%           IMP&
%             Money Market&
%             Stocks&
%             Bonds&
%             Diversified&
%             Stocks and Bonds&
%             Money Market and Stocks&
%             Others\tabularnewline
%           \hline
%           1& 14.19\%& 57.71\%& 12.21\%& 4.50\%& 7.36\%& 3.04\%& 0.99\%\tabularnewline \hline
%           2& 14.08\%& 58.18\%& 12.32\%& 4.44\%& 7.30\%& 2.80\%& 0.88\%\tabularnewline \hline
%           3 &14.26\%& 58.09\%& 12.27\%& 4.50\%& 7.19\%& 2.75\%& 0.94\%\tabularnewline \hline
%           4 &13.94\%& 58.11\%& 12.14\%& 4.78\%& 7.35\%& 2.68\%& 0.99\%\tabularnewline \hline
%           5 &13.92\%& 58.13\%& 11.93\%& 4.56\%& 7.60\%& 2.98\%& 0.88\%\tabularnewline \hline
%         \end{tabular}
%       \end{center}
%       This table presents the allocations of the wealth in the IRA
%       and Keogh accounts in various asset classes.
%       Results from each set of imputed data are presented here.
%       The first column lists the number of the imputations,
%       and rest of the columns lists various allocations.
%       Entrees under each asset class show the percentage of investors
%       who have most of their IRA
%       and Keogh wealth invested in that particular asset class.
%       The asset class Diversified
%       includes stocks,
%       bonds,
%       and money market investments.
%       The asset class Others
%       include investments in various life insurance products,
%       annuities,
%       real estate, etc.
%       \medskip
%     \footnotesize SOURCE: Survey of Consumer Finances,
%     2001,
%     Federal Reserve Board,
%     USA.\par
%   \end{table}
% }

Here is a really simple table.

% "h" means put table here---don't let it float to top or bottom of page
\begin{table}[h]
  \caption{American Presidents}
  \begin{center}
    \begin{tabular}{rl}
      \bf Number& \bf Name\\
      1& George Washington\\
      2& John Adams\\
      3& Thomas Jefferson\\
    \end{tabular}
  \end{center}
  \label{ta:American-Presidents}
\end{table}

There are 72.27 points per inch.
I like to put 2 points of vertical space between the heading
(Number Name)
and the first line
(1 George Washington)
of the table.

\begin{table}[h]
  \caption{American Presidents with 2pt vertical space after heading}
  \begin{center}
    \begin{tabular}{rl}
      \bf Number& \bf Name\\[2pt]  % put 2pt vertical space after this line
      1& George Washington\\
      2& John Adams\\
      3& Thomas Jefferson\\
    \end{tabular}
  \end{center}
  \label{ta:American-Presidents-with}
\end{table}

\LaTeX\ can print horizontal and vertical rules in tables.
I don't like the way this looks.

\begin{table}[h]
  \caption{American Presidents with horizontal and vertical lines}
  \begin{center}
    % "|" prints a vertical rule, "c" means center
    \begin{tabular}{|c|l|}
      % "\hline" prints a horizontal rule
      \hline
      \bf\#& \bf Name\\
      \hline
      1& George Washington\\
      \hline
      2& John Adams\\
      \hline
      3& Thomas Jefferson\\
      \hline
    \end{tabular}
  \end{center}
  \label{ta:American-Presidents-with}
\end{table}

\newpage

Here is a more complicated table.

\begin{table}[h]
  \caption{C Bitwise Operators}
  \begin{center}
    % "|" prints a vertical rule, "c" means center
    \begin{tabular}{cccc}
      \bf A& \bf B& \bf A$|$B& \bf A\&B\\[2pt]
      0& 0& 0& 0\\
      0& 1& 1& 0\\
      1& 0& 1& 0\\
      1& 1& 1& 1\\
    \end{tabular}
  \end{center}
  \label{ta:C-Bitwise}
\end{table}

You can use Plain \TeX's \verb+\halign+ command to make tables also.
If you can't do a complicated table using \LaTeX\ commands
you may want to try using Plain \TeX\ commands.
\LaTeX's table making commands use Plain \TeX\ commands.

\begin{table}[h]
  \caption{American Presidents using {\tt\char'134 halign}}
  \hbox to \textwidth{\hss\vbox{\halign{%
    \strut #&      % 0. \strut
    \hfil#\qquad&  % 1. Number
    #\hfil\cr      % 2. Name
    %
    & \bf Number& \bf Name\cr
    \noalign{\vskip 2pt}
    & 1& George Washington\cr
    & 2& John Adams\cr
    & 3& Thomas Jefferson\cr
  }}\hss}
  \label{ta:American-Presidents-using}
\end{table}

The next page shows how to do a table that is too long to fit on one page.

\newpage

% This is loosely based on page 106 of _A Guide to LaTeX_, third edition,
% by Helmut Kopka and Patrick W. Daly.
\begin{longtable}{|l|l|}
    \caption{State Abbreviations}\\
    \hline
    State& Abbreviation\\
    \hline
  \endfirsthead
    \caption[]{\emph{continued}}\\
    \hline
    State& Abbreviation\\
    \hline
  \endhead
    \hline
    \multicolumn{2}{r}{\emph{continued on next page}}
  \endfoot
    \hline
  \endlastfoot
  Alabama& AL\\
  Alaska& AK\\
% American Samoa& AS\\
  Arizona& AZ\\
  Arkansas& AR\\
% Armed Forces Europe& AE\\
% Armed Forces Pacific& AP\\
% Armed Forces the Americas& AA\\
  California& CA\\
  Colorado& CO\\
  Connecticut& CT\\
  Delaware& DE\\
% District of Columbia& DC\\
% Federated States of Micronesia& FM\\
  Florida& FL\\
  Georgia& GA\\
% Guam& GU\\
  Hawaii& HI\\
  Idaho& ID\\
  Illinois& IL\\
  Indiana& IN\\
  Iowa& IA\\
  Kansas& KS\\
  Kentucky& KY\\
  Louisiana& LA\\
  Maine& ME\\
% Marshall Islands& MH\\
  Maryland& MD\\
  Massachusetts& MA\\
  Michigan& MI\\
  Minnesota& MN\\
  Mississippi& MS\\
  Missouri& MO\\
  Montana& MT\\
  Nebraska& NE\\
  Nevada& NV\\
  New Hampshire& NH\\
  New Jersey& NJ\\
  New Mexico& NM\\
  New York& NY\\
  North Carolina& NC\\
  North Dakota& ND\\
% Northern Mariana Islands& MP\\
  Ohio& OH\\
  Oklahoma& OK\\
  Oregon& OR\\
  Pennsylvania& PA\\
% Puerto Rico& PR\\
  Rhode Island& RI\\
  South Carolina& SC\\
  South Dakota& SD\\
  Tennessee& TN\\
  Texas& TX\\
  Utah& UT\\
  Vermont& VT\\
% Virgin Islands& VI\\
  Virginia& VA\\
  Washington& WA\\
  West Virginia& WV\\
  Wisconsin& WI\\
  Wyoming& WY\\
\end{longtable}

\newcommand{\cbackslash}{\char'134}
\newcommand{\copencurly}{\char'173}
\newcommand{\cclosecurly}{\char'175}

\newlength{\twidth}
\newlength{\theight}

\setlength{\twidth}{\textwidth}
\setlength{\theight}{\textheight}

\begin{sidewaystable}
  % The following two lines compensate for what I think is a bug.
  \setlength{\textwidth}{\theight}
  \setlength{\textheight}{\twidth}
  \caption{%
    sidewaystable
    {\tt\cbackslash begin\copencurly tabular\cclosecurly\/}%
    \ldots
    {\tt\cbackslash end\copencurly tabular\cclosecurly\/}%
  }
  \begin{center}
    \begin{tabular}{rl}
      \bf Number& \bf Name\\[2pt]  % put 2pt vertical space after this line
      1& George Washington\\
      2& John Adams\\
      3& Thomas Jefferson\\
    \end{tabular}
  \end{center}
\end{sidewaystable}

\begin{sidewaystable}
  % The following two lines compensate for what I think is a bug.
  \setlength{\textwidth}{\theight}
  \setlength{\textheight}{\twidth}
  \caption{%
    sidewaystable
    {\tt\cbackslash halign\copencurly}\ldots{\tt\cclosecurly\/} table%
  }
  \hbox to \textwidth{\hss\vbox{\halign{%
    \strut #&      % 0. \strut
    \hfil#\qquad&  % 1. Number
    #\hfil\cr      % 2. Name
    %
    & \bf Number& \bf Name\cr
    \noalign{\vskip 2pt}
    & 1& George Washington\cr
    & 2& John Adams\cr
    & 3& Thomas Jefferson\cr
  }}\hss}
\end{sidewaystable}

%\newlength{\ta}
%\settowidth{\ta}{\vbox{\hbox{Money}\hbox{Market}}}
%\newlength{\tb}
%\settowidth{\tb}{\vbox{\hbox{Stocks}\hbox{and}\hbox{Bonds}}}
%\newlength{\tc}
%\settowidth{\tc}{\vbox{\hbox{Money}\hbox{Market}\hbox{and}\hbox{Stocks}}}
%
%  {\renewcommand{\baselinestretch}{1}
%\begin{table}
%  \caption{\hfil Allocation of the IRA and Keogh Wealth\hfil\break\mbox{}\hfil for Investors With or Without Brokerage Accounts\hfil}
%  \label{tab:ira}
%  \begin{center}
%    \begin{tabular}%
%      {%
%        |%
%        c%
%        |%
%        >{\centering\hspace{0pt}}m{\the\ta}%  Money Market
%        |%
%        c%                                    Stocks 
%        |%
%        c%                                    Bonds
%        |%
%        c%                                    Diversified
%        |%
%        >{\centering\hspace{0pt}}m{\the\tb}%  Stocks and Bonds
%        |%
%        >{\centering\hspace{0pt}}m{\the\tc}%  Money Market and Stocks
%        |%
%        c%                                    Others
%        |%
%      }
%      \hline
%      IMP&
%        Money Market&
%        Stocks&
%        Bonds&
%        Diversified&
%        Stocks and Bonds&
%        Money Market and Stocks&
%        Others\tabularnewline
%      \hline
%      1& 14.19\%& 57.71\%& 12.21\%& 4.50\%& 7.36\%& 3.04\%& 0.99\%\tabularnewline \hline
%      2& 14.08\%& 58.18\%& 12.32\%& 4.44\%& 7.30\%& 2.80\%& 0.88\%\tabularnewline \hline
%      3 &14.26\%& 58.09\%& 12.27\%& 4.50\%& 7.19\%& 2.75\%& 0.94\%\tabularnewline \hline
%      4 &13.94\%& 58.11\%& 12.14\%& 4.78\%& 7.35\%& 2.68\%& 0.99\%\tabularnewline \hline
%      5 &13.92\%& 58.13\%& 11.93\%& 4.56\%& 7.60\%& 2.98\%& 0.88\%\tabularnewline \hline
%    \end{tabular}
%  \end{center}
%  This table presents the allocations of the wealth in the IRA
%  and Keogh accounts in various asset classes.
%  Results from each set of imputed data are presented here.
%  The first column lists the number of the imputations,
%  and rest of the columns lists various allocations.
%  Entrees under each asset class show the percentage of investors
%  who have most of their IRA
%  and Keogh wealth invested in that particular asset class.
%  The asset class Diversified
%  includes stocks,
%  bonds,
%  and money market investments.
%  The asset class Others
%  include investments in various life insurance products,
%  annuities,
%  real estate, etc.
%  \medskip
%  \footnotesize SOURCE: Survey of Consumer Finances,
%  2001,
%  Federal Reserve Board,
%  USA.\par
%\end{table}
%  }


%
%  demo-text.tex  2007-07-17  Mark Senn  http://engineering.purdue.edu/~mark
%

\chapter{Demonstrate Text}

% Use single spacing.
\Baselinestretch{1}

% You don't normally need this.
\mbox{}


%\vbox{
\begin{verbatim}
This is a sentence.
This is a sentence.
This is a sentence.
This is a sentence.
This is a sentence.

This is a sentence.
This is a sentence.
This is a sentence.
This is a sentence.
This is a sentence.
\end{verbatim}
This is a sentence.
This is a sentence.
This is a sentence.
This is a sentence.
This is a sentence.

This is a sentence.
This is a sentence.
This is a sentence.
This is a sentence.
This is a sentence.
\vskip\baselineskip
\hrule
%}
\vskip0.5\baselineskip
\filbreak

%\vbox{
\begin{verbatim}
From \verb+http://www.biblegateway.com/passage/?book_id=1&chapter=1&version=50+:

\begin{quote}
    1 In the beginning God created the heavens and the earth.
    2 The earth was without form,
    and void;
    and darkness was on the face of the deep.
    And the Spirit of God was hovering over the face of the waters.

    3 Then God said,``Let there be light'';
    and there was light.
    4 And God saw the light,
    that it was good;
    and God divided the light from the darkness.
    5 God called the light Day,
    and the darkness He called Night.
    So the evening and the morning were the first day. 
\end{quote}
\end{verbatim}
From \verb+http://www.biblegateway.com/passage/?book_id=1&chapter=1&version=50+:

\begin{quote}
    1 In the beginning God created the heavens and the earth.
    2 The earth was without form,
    and void;
    and darkness was on the face of the deep.
    And the Spirit of God was hovering over the face of the waters.

    3 Then God said,``Let there be light'';
    and there was light.
    4 And God saw the light,
    that it was good;
    and God divided the light from the darkness.
    5 God called the light Day,
    and the darkness He called Night.
    So the evening and the morning were the first day. 
\end{quote}
\vskip\baselineskip
\hrule
%}
\vskip0.5\baselineskip
\filbreak

%\vbox{
\begin{verbatim}
\begin{description}
    \item[apple]
        A red fruit.
    \item[banana]
        A yellow fruit.
        This sentence is to make the entry longer so you can see what happens.
        This sentence is to make the entry longer so you can see what happens.
    \item[cherry]
        A red friut.
\end{description}
\end{verbatim}
\begin{description}
    \item[apple]
        A red fruit.
    \item[banana]
        A yellow fruit.
        This sentence is to make the entry longer so you can see what happens.
        This sentence is to make the entry longer so you can see what happens.
    \item[cherry]
        A red friut.
\end{description}
\vskip\baselineskip
\hrule
%}
\vskip0.5\baselineskip
\filbreak

%\vbox{
\begin{verbatim}
\begin{enumerate}
    \item apple
    \item banana
        This sentence is to make the entry longer so you can see what happens.
        This sentence is to make the entry longer so you can see what happens.
    \item cherry
\end{enumerate}
\end{verbatim}
\begin{enumerate}
    \item apple
    \item banana
        This sentence is to make the entry longer so you can see what happens.
        This sentence is to make the entry longer so you can see what happens.
    \item cherry
\end{enumerate}
\vskip\baselineskip
\hrule
%}
\vskip0.5\baselineskip
\filbreak


%\vbox{
\begin{verbatim}
\begin{itemize}
    \item apple
    \item banana
        This sentence is to make the entry longer so you can see what happens.
        This sentence is to make the entry longer so you can see what happens.
    \item cherry
\end{itemize}
\end{verbatim}
\begin{itemize}
    \item apple
    \item banana
        This sentence is to make the entry longer so you can see what happens.
        This sentence is to make the entry longer so you can see what happens.
    \item cherry
\end{itemize}
\vskip\baselineskip
\hrule
%}
\vskip0.5\baselineskip
\filbreak


% Bibliography is required if you consulted any outside references.
% Reference: TM 32.
\include{bibliography}

% Notes and footnotes are optional.
% Reference: TM 34.
% I have not implemented this yet.  Mark Senn 2002-06-03
%%\include{notes}

% A vita is optional for masters theses
% and required for doctoral dissertations.
% Reference: TM 13.
% CHANGE NEXT LINE?
%  A vita is required only in a doctoral dissertation.
%

\begin{vita}
    [Put a brief autobiographical sketch here.]
\end{vita}


\end{document}

% LaTeX won't read after the \end{document} command.
% You can put notes to yourself or LaTeX input not
% ready for use here if you'd like.
