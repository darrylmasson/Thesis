% front matter

\begin{statement}
    \entry{Dr.~Rafael F. Lang, Chair}{Department of Physics and Astronomy}
    \entry{Dr.~John Finley}{Department of Physics and Astronomy}
    \entry{Dr.~Ephraim Fishbach}{Department of Physics and Astronomy}
    \entry{Dr.~Matthew Jones}{Department of Physics and Astronomy}
    \approvedby{Dr.~John Finley}{Head of the Department of Physics and Astronomy}
\end{statement}

  % The Table of Contents will be automatically created for you
  % using information you supply in
  %     \chapter
  %     \section
  %     \subsection
  %     \subsubsection
  % commands.
\tableofcontents

  % The List of Tables will be automatically created for you using
  % information you supply in
  %     \begin{table} ... \end{table}
  % environments.
\listoftables

  % The List of Figures will be automatically created for you using
  % information you supply in
  %     \begin{figure} ... \end{figure}
  % environments.
\listoffigures

  % List of Abbreviations is optional.
\begin{abbreviations}

DM& Dark matter\cr
GXe& Gasseous xenon\cr
LNGS& Laboratori Nazionali del Gran Sasso\cr
LXe& Liquid xenon\cr
PMT& Photomultiplier tube\cr
PTFE& Polytetrafluoroethylene\cr
ROI& Region of Interest\cr
SR0& Science run 0 (2016.10-2017.01)\cr
SR1& Science run 1 (2017.02-2018.02)\cr
TPC& Time projection chamber\cr
WIMP& Weakly interacting massive particle\cr

\end{abbreviations}

  % Abstract is required.

\begin{abstract}

As dark matter detection experiments continue to report null results, the need for larger and more sensitive detectors means even more stringent design requirements. New calibration techniques are required and better calibration methods become possible with increased detector size. Additionally, previously ignored detector features such as convection become important, especially as internal, dissolvable sources become more common. Furthermore, convection also offers the possibility for reduction of the $^{222}$Rn backrounds via an offline analysis where atoms of $^{214}$Pb are tagged and followed throughout the detector via a technique dubbed the ``radon self-veto''. In this thesis, we present the characterization of a deuterium-deuterium plasma fusion neutron generator optimized to perform the nuclear recoil calibration of XENON1T. Part of this characterization is done with liquid organic scintillator detectors, which are sensitive to both electonic and nuclear recoil interactions. We develop a new algorithm for discriminating between these two signal types using Laplace transforms and show that it performs better than traditional algorithms. A multipurpose source of dissolvable $^{220}$Rn is presented and measurements made of long-lived contaminants from this source. Finally, we present the first measurement of convection in XENON1T and report the results of a simple convection-agnostic implementation of the radon self-veto.

\end{abstract}
