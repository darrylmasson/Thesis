
% chapter two: searching for dark matter

\chapter{The Search for Dark Matter}

\section{Types of search}

%Now that a promising model of dark matter have been proposed, some time should be spent discussing how it might be tested. There are three classes of experiments to detect WIMPs. One can try to produce them by smashing known standard model particles together. Or, one can try to observe two WIMPs smashing into each other and annihilating. Thirdly, one can try to observe a WIMP directly interacting with some standard model particle. We will discuss these three methods, their limitations, and their strengths.

\subsection{Production Searches}

%Were one to take two standard model particles, give them together with gratuitous amounts of energy, and and smash them into one another, one generally produces a large menagerie of other particles that tend to decay fairly quickly. However, assuming there is some nonzero coupling between dark matter and the standard model, there is a probability that instead of producing other standard model particles, the collision will produce dark matter which will leave the detector without causing it to trigger. The signature of this will be missing energy and momentum from the collision. The goal, then, is to search through reams of data for these signatures. However, if there isn't a trigger there is no evidence that anything happened in the detector. This means one has to look for cases where both dark matter and standard model particles are created (so something can cause a trigger), but each additional coupling in the Feynman diagram is accompanied by a reduced probability. As the strength of these couplings is not known, this process is a little like searching for a needle in a haystack when you don't know what a needle looks like.

\subsection{Annihilation Searches}

%Given the density of dark matter observed today, there is some finite probability that two dark matter particles will run into each other with sufficient energy to annihilate. If we happen to be fortunate enough to be pointing a telescope towards this event at the appropriate time, its signature could be detected. The goal here is to filter out which events are from known astrophysical sources and which are the dark matter signals. This is never straightforward. One could compare it to searching for a needle in a stack of slightly different needles. On numerous occasions, peaks have shown up in the data to some significance, but then vanish once a complementary experiment begins collecting data in that region of the spectrum. Additionally, the astrophysical sources are not yet completely understood.

\subsection{Scattering Searches}

%We can continue to rotate our Feynman graph to demonstrate a scattering event. In a scattering event, the interaction between the dark matter particle and the target standard model particle will cause the target to recoil. The infomation generated in this event will be contained in the ionization of the target atom, its resulting scintillation, and the creation of phonons as the nucleus recoils into its neighboring atoms. Depending on the specifics of the detector, some combination of these can be collected. The two leading technologies in this field are based on germanium crystals and liquid noble time projection chambers (TPCs).

\section{Fundamentals of direct detection}

\subsection{Energy spectrum}

\subsection{Annual modulation}

\subsection{Backgrounds}